\documentclass{classrep}
\usepackage[utf8]{inputenc}
\usepackage{color}

\studycycle{Informatyka, studia dzienne, I st.}
\coursesemester{VI}

\coursename{Komputerowe systemy rozpoznawania}
\courseyear{2018/2019}

\courseteacher{dr inż. Marcin Kacprowicz}
\coursegroup{poniedziałek, 14.10}

\author{
	\studentinfo{Justyna Hubert}{210200} \and
	\studentinfo{Karol Podlewski}{210294}
}

\title{Zadanie 1: Ekstrakcja cech, miary podobieństwa, klasyfikacja}
\svnurl{https://github.com/hubjust/KSR}

\begin{document}
	\maketitle
	
	
	\section{Cel}
	{\color{blue} 
		W tej sekcji należy zamieścić zwięzły (maksymalnie dwa, trzy zdania) opis
		problemu, który był rozwiązywany (uwzględnić należy zarówno część badawczą jak
		i implementacyjną).}
	
	\section{Wprowadzenie}
	{\color{blue}
		We wprowadzeniu należy zaprezentować całą teorię potrzebną do realizacji
		zadania (przy czym należy tu ograniczyć się wyłącznie do tego, co było
		wykorzystane) tak aby osoba, która nigdy wcześniej nie zetknęła się z tą
		tematyką, potrafiła zrozumieć dalszy opis. Część ta powinna wprowadzać
		wszystkie wykorzystywane wzory, oznaczenia itp., do których należy się
		odwoływać w dalszej części niniejszgo sprawozdania. Zamieszczony tu własny
		opis teorii (a nie skopiowany!) należy poprzeć odwołaniami bibliograficznymi
		do literatury zamieszczonej na końcu. }
	
	\section{Opis implementacji}
	{\color{blue}
		Należy tu zamieścić krótki i zwięzły opis zaprojektowanych klas oraz powiązań
		między nimi. Powinien się tu również znaleźć diagram UML  (diagram klas)
		prezentujący najistotniejsze elementy stworzonej aplikacji. Należy także
		podać, w jakim języku programowania została stworzona aplikacja. }
	
	\section{Materiały i metody}
	{\color{blue}
		W tym miejscu należy opisać, jak przeprowadzone zostały wszystkie badania,
		których wyniki i dyskusja zamieszczane są w dalszych sekcjach. Opis ten
		powinien być na tyle dokładny, aby osoba czytająca go potrafiła wszystkie
		przeprowadzone badania samodzielnie powtórzyć w celu zweryfikowania ich
		poprawności (a zatem m.in. należy zamieścić tu opis architektury sieci,
		wartości współczynników użytych w kolejnych eksperymentach, sposób
		inicjalizacji wag, metodę uczenia itp. oraz informacje o danych, na których
		prowadzone były badania). Przy opisie należy odwoływać się i stosować do
		opisanych w sekcji drugiej wzorów i oznaczeń, a także w jasny sposób opisać
		cel konkretnego testu. Najlepiej byłoby wyraźnie wyszczególnić (ponumerować)
		poszczególne eksperymenty tak, aby łatwo było się do nich odwoływać dalej.}
	
	\section{Wyniki}
	{\color{blue}
		W tej sekcji należy zaprezentować, dla każdego przeprowadzonego eksperymentu,
		kompletny zestaw wyników w postaci tabel, wykresów itp. Powinny być one tak
		ponazywane, aby było wiadomo, do czego się odnoszą. Wszystkie tabele i wykresy
		należy oczywiście opisać (opisać co jest na osiach, w kolumnach itd.) stosując
		się do przyjętych wcześniej oznaczeń. Nie należy tu komentować i interpretować
		wyników, gdyż miejsce na to jest w kolejnej sekcji. Tu również dobrze jest
		wprowadzić oznaczenia (tabel, wykresów) aby móc się do nich odwoływać
		poniżej.}
	
	\section{Dyskusja}
	{\color{blue}
		Sekcja ta powinna zawierać dokładną interpretację uzyskanych wyników
		eksperymentów wraz ze szczegółowymi wnioskami z nich płynącymi. Najcenniejsze
		są, rzecz jasna, wnioski o charakterze uniwersalnym, które mogą być istotne
		przy innych, podobnych zadaniach. Należy również omówić i wyjaśnić wszystkie
		napotakane problemy (jeśli takie były). Każdy wniosek powinien mieć poparcie
		we wcześniej przeprowadzonych eksperymentach (odwołania do konkretnych
		wyników). Jest to jedna z najważniejszych sekcji tego sprawozdania, gdyż
		prezentuje poziom zrozumienia badanego problemu.}
	\section{Wnioski}
	{\color{blue}W tej, przedostatniej, sekcji należy zamieścić podsumowanie
		najważniejszych wniosków z sekcji poprzedniej. Najlepiej jest je po prostu
		wypunktować. Znów, tak jak poprzednio, najistotniejsze są wnioski o
		charakterze uniwersalnym.}
	
	
	\begin{thebibliography}{}
		\bibitem{adam}
		Methods for the linguistic summarization of data - aplications of fuzzy sets and their extensions, Adam Niewiadomski, Akademicka Oficyna Wydawnicza EXIT, Warszawa 2008
	\end{thebibliography}
\end{document}
